\section{Modes for reading database file}
\label{group__read__modes}\index{Modes for reading database file@{Modes for reading database file}}
\subsection*{Defines}
\begin{CompactItemize}
\item 
\#define {\bf READ\_\-MODE\_\-MERGE\_\-NO\_\-MERGE}\ 0
\item 
\#define {\bf READ\_\-MODE\_\-REPORT\_\-NO\_\-MERGE}\ 1
\item 
\#define {\bf READ\_\-MODE\_\-MERGE\_\-INST\_\-MERGE}\ 2
\item 
\#define {\bf READ\_\-MODE\_\-REPORT\_\-MOD\_\-MERGE}\ 3
\end{CompactItemize}


\subsection{Detailed Description}
Specify how to integrate read data from database file into memory structures. 

\subsection{Define Documentation}
\index{read_modes@{read\_\-modes}!READ_MODE_MERGE_INST_MERGE@{READ\_\-MODE\_\-MERGE\_\-INST\_\-MERGE}}
\index{READ_MODE_MERGE_INST_MERGE@{READ\_\-MODE\_\-MERGE\_\-INST\_\-MERGE}!read_modes@{read\_\-modes}}
\subsubsection{\setlength{\rightskip}{0pt plus 5cm}\#define READ\_\-MODE\_\-MERGE\_\-INST\_\-MERGE\ 2}\label{group__read__modes_a2}


When module is completely read in (including module, signals, expressions), the module is looked up in the current instance tree. If the instance exists, the module is merged with the instance's module; otherwise, we are attempting to merge two databases that were created from differe9nt designs. \index{read_modes@{read\_\-modes}!READ_MODE_MERGE_NO_MERGE@{READ\_\-MODE\_\-MERGE\_\-NO\_\-MERGE}}
\index{READ_MODE_MERGE_NO_MERGE@{READ\_\-MODE\_\-MERGE\_\-NO\_\-MERGE}!read_modes@{read\_\-modes}}
\subsubsection{\setlength{\rightskip}{0pt plus 5cm}\#define READ\_\-MODE\_\-MERGE\_\-NO\_\-MERGE\ 0}\label{group__read__modes_a0}


When new module is read from database file, it is placed in the module list and is placed in the correct hierarchical position in the instance tree. Used when performing a MERGE command. \index{read_modes@{read\_\-modes}!READ_MODE_REPORT_MOD_MERGE@{READ\_\-MODE\_\-REPORT\_\-MOD\_\-MERGE}}
\index{READ_MODE_REPORT_MOD_MERGE@{READ\_\-MODE\_\-REPORT\_\-MOD\_\-MERGE}!read_modes@{read\_\-modes}}
\subsubsection{\setlength{\rightskip}{0pt plus 5cm}\#define READ\_\-MODE\_\-REPORT\_\-MOD\_\-MERGE\ 3}\label{group__read__modes_a3}


When module is completely read in (including module, signals, expressions), the module is looked up in the module list. If the module is found, it is merged with the existing module; otherwise, it is added to the module list. \index{read_modes@{read\_\-modes}!READ_MODE_REPORT_NO_MERGE@{READ\_\-MODE\_\-REPORT\_\-NO\_\-MERGE}}
\index{READ_MODE_REPORT_NO_MERGE@{READ\_\-MODE\_\-REPORT\_\-NO\_\-MERGE}!read_modes@{read\_\-modes}}
\subsubsection{\setlength{\rightskip}{0pt plus 5cm}\#define READ\_\-MODE\_\-REPORT\_\-NO\_\-MERGE\ 1}\label{group__read__modes_a1}


When new module is read from database file, it is placed in the module list and is placed in the correct hierarchical position in the instance tree. Used when performing a REPORT command. 