\section{merge.c File Reference}
\label{merge_8c}\index{merge.c@{merge.c}}
{\tt \#include $<$stdio.h$>$}\par
{\tt \#include $<$stdlib.h$>$}\par
{\tt \#include \char`\"{}db.h\char`\"{}}\par
{\tt \#include \char`\"{}defines.h\char`\"{}}\par
{\tt \#include \char`\"{}merge.h\char`\"{}}\par
{\tt \#include \char`\"{}util.h\char`\"{}}\par
\subsection*{Functions}
\begin{CompactItemize}
\item 
void {\bf merge\_\-usage} ()
\item 
{\bf bool} {\bf merge\_\-parse\_\-args} (int argc, int last\_\-arg, char $\ast$$\ast$argv)
\item 
int {\bf command\_\-merge} (int argc, int last\_\-arg, char $\ast$$\ast$argv)
\begin{CompactList}\small\item\em Parses command-line for merge options and performs merge command.\item\end{CompactList}\end{CompactItemize}
\subsection*{Variables}
\begin{CompactItemize}
\item 
char $\ast$ {\bf merged\_\-file} = NULL
\item 
char $\ast$ {\bf merge\_\-in0} = NULL
\item 
char $\ast$ {\bf merge\_\-in1} = NULL
\item 
char {\bf user\_\-msg} [USER\_\-MSG\_\-LENGTH]
\end{CompactItemize}


\subsection{Detailed Description}
\begin{Desc}
\item[Author:]Trevor Williams ({\tt trevorw@charter.net}) \end{Desc}
\begin{Desc}
\item[Date:]11/29/2001\end{Desc}


\subsection{Function Documentation}
\index{merge.c@{merge.c}!command_merge@{command\_\-merge}}
\index{command_merge@{command\_\-merge}!merge.c@{merge.c}}
\subsubsection{\setlength{\rightskip}{0pt plus 5cm}int command\_\-merge (int {\em argc}, int {\em last\_\-arg}, char $\ast$$\ast$ {\em argv})}\label{merge_8c_a6}


Parses command-line for merge options and performs merge command.

\begin{Desc}
\item[Parameters:]
\begin{description}
\item[{\em argc}]Number of arguments in command-line to parse. \item[{\em last\_\-arg}]Index of last parsed argument from list. \item[{\em argv}]List of arguments from command-line to parse.\end{description}
\end{Desc}
\begin{Desc}
\item[Returns:]Returns 0 if merge is successful; otherwise, returns -1.\end{Desc}
Performs merge command functionality. \index{merge.c@{merge.c}!merge_parse_args@{merge\_\-parse\_\-args}}
\index{merge_parse_args@{merge\_\-parse\_\-args}!merge.c@{merge.c}}
\subsubsection{\setlength{\rightskip}{0pt plus 5cm}{\bf bool} merge\_\-parse\_\-args (int {\em argc}, int {\em last\_\-arg}, char $\ast$$\ast$ {\em argv})}\label{merge_8c_a5}


\begin{Desc}
\item[Parameters:]
\begin{description}
\item[{\em argc}]Number of arguments in argument list argv. \item[{\em last\_\-arg}]Index of last parsed argument from list. \item[{\em argv}]Argument list passed to this program.\end{description}
\end{Desc}
\begin{Desc}
\item[Returns:]Returns TRUE if argument parsing was successful; otherwise, returns FALSE.\end{Desc}
Parses the merge argument list, placing all parsed values into global variables. If an argument is found that is not valid for the merge operation, an error message is displayed to the user. \index{merge.c@{merge.c}!merge_usage@{merge\_\-usage}}
\index{merge_usage@{merge\_\-usage}!merge.c@{merge.c}}
\subsubsection{\setlength{\rightskip}{0pt plus 5cm}void merge\_\-usage ()}\label{merge_8c_a4}


Outputs usage informaiton to standard output for merge command. 

\subsection{Variable Documentation}
\index{merge.c@{merge.c}!merge_in0@{merge\_\-in0}}
\index{merge_in0@{merge\_\-in0}!merge.c@{merge.c}}
\subsubsection{\setlength{\rightskip}{0pt plus 5cm}char$\ast$ merge\_\-in0 = NULL}\label{merge_8c_a1}


\index{merge.c@{merge.c}!merge_in1@{merge\_\-in1}}
\index{merge_in1@{merge\_\-in1}!merge.c@{merge.c}}
\subsubsection{\setlength{\rightskip}{0pt plus 5cm}char$\ast$ merge\_\-in1 = NULL}\label{merge_8c_a2}


\index{merge.c@{merge.c}!merged_file@{merged\_\-file}}
\index{merged_file@{merged\_\-file}!merge.c@{merge.c}}
\subsubsection{\setlength{\rightskip}{0pt plus 5cm}char$\ast$ merged\_\-file = NULL}\label{merge_8c_a0}


\index{merge.c@{merge.c}!user_msg@{user\_\-msg}}
\index{user_msg@{user\_\-msg}!merge.c@{merge.c}}
\subsubsection{\setlength{\rightskip}{0pt plus 5cm}char user\_\-msg[USER\_\-MSG\_\-LENGTH] ()}\label{merge_8c_a3}


Holds some output that will be displayed via the print\_\-output command. This is created globally so that memory does not need to be reallocated for each function that wishes to use it. 