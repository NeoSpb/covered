\section{score.c File Reference}
\label{score_8c}\index{score.c@{score.c}}
{\tt \#include $<$stdlib.h$>$}\par
{\tt \#include $<$stdio.h$>$}\par
{\tt \#include $<$string.h$>$}\par
{\tt \#include \char`\"{}defines.h\char`\"{}}\par
{\tt \#include \char`\"{}score.h\char`\"{}}\par
{\tt \#include \char`\"{}util.h\char`\"{}}\par
{\tt \#include \char`\"{}link.h\char`\"{}}\par
{\tt \#include \char`\"{}search.h\char`\"{}}\par
{\tt \#include \char`\"{}parse.h\char`\"{}}\par
\subsection*{Functions}
\begin{CompactItemize}
\item 
void {\bf score\_\-usage} ()
\item 
{\bf bool} {\bf read\_\-command\_\-file} (char $\ast$cmd\_\-file, char $\ast$$\ast$arg\_\-list, int $\ast$arg\_\-num)
\item 
{\bf bool} {\bf score\_\-parse\_\-args} (int argc, char $\ast$$\ast$argv)
\item 
int {\bf command\_\-score} (int argc, char $\ast$$\ast$argv)
\begin{CompactList}\small\item\em Parses score command-line and performs score.\item\end{CompactList}\end{CompactItemize}
\subsection*{Variables}
\begin{CompactItemize}
\item 
char $\ast$ {\bf top\_\-module} = NULL
\item 
char $\ast$ {\bf output\_\-db} = NULL
\item 
char $\ast$ {\bf vcd\_\-file} = NULL
\item 
unsigned long {\bf largest\_\-malloc\_\-size}
\item 
unsigned long {\bf curr\_\-malloc\_\-size}
\end{CompactItemize}


\subsection{Detailed Description}


\begin{Desc}
\item[Author: ]\par
Trevor Williams ({\tt trevorw@charter.net}) \end{Desc}
\begin{Desc}
\item[Date: ]\par
11/29/2001\end{Desc}


\subsection{Function Documentation}
\index{score.c@{score.c}!command_score@{command\_\-score}}
\index{command_score@{command\_\-score}!score.c@{score.c}}
\subsubsection{\setlength{\rightskip}{0pt plus 5cm}int command\_\-score (int {\em argc}, char $\ast$$\ast$ {\em argv})}\label{score_8c_a8}


Parses score command-line and performs score.

\begin{Desc}
\item[Parameters: ]\par
\begin{description}
\item[{\em 
argc}]Number of arguments in score command-line. \item[{\em 
argv}]Arguments from command-line to parse. \end{description}
\end{Desc}
\begin{Desc}
\item[Returns: ]\par
Returns 0 if scoring is successful; otherwise, returns -1. \end{Desc}
\index{score.c@{score.c}!read_command_file@{read\_\-command\_\-file}}
\index{read_command_file@{read\_\-command\_\-file}!score.c@{score.c}}
\subsubsection{\setlength{\rightskip}{0pt plus 5cm}{\bf bool} read\_\-command\_\-file (char $\ast$ {\em cmd\_\-file}, char $\ast$$\ast$ {\em arg\_\-list}, int $\ast$ {\em arg\_\-num})}\label{score_8c_a6}


\begin{Desc}
\item[Parameters: ]\par
\begin{description}
\item[{\em 
cmd\_\-file}]Name of file to read commands from. \item[{\em 
arg\_\-list}]List of arguments found in specified command file. \item[{\em 
arg\_\-num}]Number of arguments in arg\_\-list array. \end{description}
\end{Desc}
\begin{Desc}
\item[Returns: ]\par
Returns TRUE if read of command file was successful; otherwise, returns FALSE. \end{Desc}
\index{score.c@{score.c}!score_parse_args@{score\_\-parse\_\-args}}
\index{score_parse_args@{score\_\-parse\_\-args}!score.c@{score.c}}
\subsubsection{\setlength{\rightskip}{0pt plus 5cm}{\bf bool} score\_\-parse\_\-args (int {\em argc}, char $\ast$$\ast$ {\em argv})}\label{score_8c_a7}


\begin{Desc}
\item[Parameters: ]\par
\begin{description}
\item[{\em 
argc}]Number of arguments specified in argv parameter list. \item[{\em 
argv}]List of arguments to parse. \end{description}
\end{Desc}
\begin{Desc}
\item[Returns: ]\par
Returns TRUE if successful in dealing with arguments; otherwise, returns FALSE. \end{Desc}
\index{score.c@{score.c}!score_usage@{score\_\-usage}}
\index{score_usage@{score\_\-usage}!score.c@{score.c}}
\subsubsection{\setlength{\rightskip}{0pt plus 5cm}void score\_\-usage ()}\label{score_8c_a5}


Displays usage information for score command. 

\subsection{Variable Documentation}
\index{score.c@{score.c}!curr_malloc_size@{curr\_\-malloc\_\-size}}
\index{curr_malloc_size@{curr\_\-malloc\_\-size}!score.c@{score.c}}
\subsubsection{\setlength{\rightskip}{0pt plus 5cm}unsigned long curr\_\-malloc\_\-size ()}\label{score_8c_a4}


Contains the total number of bytes malloc'ed during the simulation run. This information is output to the user after simulation as a performance indicator. \index{score.c@{score.c}!largest_malloc_size@{largest\_\-malloc\_\-size}}
\index{largest_malloc_size@{largest\_\-malloc\_\-size}!score.c@{score.c}}
\subsubsection{\setlength{\rightskip}{0pt plus 5cm}unsigned long largest\_\-malloc\_\-size ()}\label{score_8c_a3}


Holds the largest number of bytes in allocation at one period of time. \index{score.c@{score.c}!output_db@{output\_\-db}}
\index{output_db@{output\_\-db}!score.c@{score.c}}
\subsubsection{\setlength{\rightskip}{0pt plus 5cm}char$\ast$ output\_\-db = NULL}\label{score_8c_a1}


Name of output score database file to generate \index{score.c@{score.c}!top_module@{top\_\-module}}
\index{top_module@{top\_\-module}!score.c@{score.c}}
\subsubsection{\setlength{\rightskip}{0pt plus 5cm}char$\ast$ top\_\-module = NULL}\label{score_8c_a0}


Name of top-level module to score \index{score.c@{score.c}!vcd_file@{vcd\_\-file}}
\index{vcd_file@{vcd\_\-file}!score.c@{score.c}}
\subsubsection{\setlength{\rightskip}{0pt plus 5cm}char$\ast$ vcd\_\-file = NULL}\label{score_8c_a2}


Name of VCD output file to parse 