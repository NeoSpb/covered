\section{expr.c File Reference}
\label{expr_8c}\index{expr.c@{expr.c}}
{\tt \#include $<$stdio.h$>$}\par
{\tt \#include $<$stdlib.h$>$}\par
{\tt \#include $<$assert.h$>$}\par
{\tt \#include \char`\"{}defines.h\char`\"{}}\par
{\tt \#include \char`\"{}expr.h\char`\"{}}\par
{\tt \#include \char`\"{}link.h\char`\"{}}\par
{\tt \#include \char`\"{}vector.h\char`\"{}}\par
{\tt \#include \char`\"{}binding.h\char`\"{}}\par
{\tt \#include \char`\"{}util.h\char`\"{}}\par
{\tt \#include \char`\"{}sim.h\char`\"{}}\par
\subsection*{Functions}
\begin{CompactItemize}
\item 
void {\bf expression\_\-create\_\-value} ({\bf expression} $\ast$exp, int width, int lsb, {\bf bool} data)
\begin{CompactList}\small\item\em Creates an expression value and initializes it.\item\end{CompactList}\item 
{\bf expression} $\ast$ {\bf expression\_\-create} ({\bf expression} $\ast$right, {\bf expression} $\ast$left, int op, int id, int line, {\bf bool} data)
\begin{CompactList}\small\item\em Creates new expression.\item\end{CompactList}\item 
void {\bf expression\_\-set\_\-value} ({\bf expression} $\ast$exp, {\bf vector} $\ast$vec)
\begin{CompactList}\small\item\em Sets the specified expression value to the specified vector value.\item\end{CompactList}\item 
void {\bf expression\_\-resize} ({\bf expression} $\ast$expr, {\bf bool} recursive)
\begin{CompactList}\small\item\em Recursively resizes specified expression tree leaf node.\item\end{CompactList}\item 
int {\bf expression\_\-get\_\-id} ({\bf expression} $\ast$expr)
\begin{CompactList}\small\item\em Returns expression ID of this expression.\item\end{CompactList}\item 
void {\bf expression\_\-db\_\-write} ({\bf expression} $\ast$expr, FILE $\ast${\bf file}, char $\ast$scope)
\begin{CompactList}\small\item\em Writes this expression to the specified database file.\item\end{CompactList}\item 
{\bf bool} {\bf expression\_\-db\_\-read} (char $\ast$$\ast$line, {\bf module} $\ast$curr\_\-mod, {\bf bool} eval)
\begin{CompactList}\small\item\em Reads current line of specified file and parses for expression information.\item\end{CompactList}\item 
{\bf bool} {\bf expression\_\-db\_\-merge} ({\bf expression} $\ast$base, char $\ast$$\ast$line, {\bf bool} same)
\begin{CompactList}\small\item\em Reads and merges two expressions and stores result in base expression.\item\end{CompactList}\item 
void {\bf expression\_\-display} ({\bf expression} $\ast$expr)
\begin{CompactList}\small\item\em Displays the specified expression information.\item\end{CompactList}\item 
void {\bf expression\_\-operate} ({\bf expression} $\ast$expr)
\begin{CompactList}\small\item\em Performs operation specified by parameter expression.\item\end{CompactList}\item 
void {\bf expression\_\-operate\_\-recursively} ({\bf expression} $\ast$expr)
\begin{CompactList}\small\item\em Performs recursive expression operation (parse mode only).\item\end{CompactList}\item 
int {\bf expression\_\-bit\_\-value} ({\bf expression} $\ast$expr)
\begin{CompactList}\small\item\em Returns a compressed, 1-bit representation of the value after a unary OR.\item\end{CompactList}\item 
{\bf bool} {\bf expression\_\-is\_\-static\_\-only} ({\bf expression} $\ast$expr)
\begin{CompactList}\small\item\em Returns TRUE if specified expression is found to contain all static leaf expressions.\item\end{CompactList}\item 
void {\bf expression\_\-dealloc} ({\bf expression} $\ast$expr, {\bf bool} exp\_\-only)
\begin{CompactList}\small\item\em Deallocates memory used for expression.\item\end{CompactList}\end{CompactItemize}
\subsection*{Variables}
\begin{CompactItemize}
\item 
{\bf nibble} {\bf xor\_\-optab} [16]
\item 
{\bf nibble} {\bf and\_\-optab} [16]
\item 
{\bf nibble} {\bf or\_\-optab} [16]
\item 
{\bf nibble} {\bf nand\_\-optab} [16]
\item 
{\bf nibble} {\bf nor\_\-optab} [16]
\item 
{\bf nibble} {\bf nxor\_\-optab} [16]
\item 
int {\bf curr\_\-sim\_\-time}
\item 
char {\bf user\_\-msg} [USER\_\-MSG\_\-LENGTH]
\item 
{\bf exp\_\-link} $\ast$ {\bf static\_\-expr\_\-head}
\item 
{\bf exp\_\-link} $\ast$ {\bf static\_\-expr\_\-tail}
\end{CompactItemize}


\subsection{Detailed Description}
\begin{Desc}
\item[Author:]Trevor Williams ({\tt trevorw@charter.net}) \end{Desc}
\begin{Desc}
\item[Date:]12/1/2001\end{Desc}
\begin{Desc}
\item[Expressions]The following are special expressions that are handled differently than standard unary (i.e., $\sim$a) and dual operators (i.e., a \& b). These expressions are documented to help remove confusion (my own) about how they are implemented by Covered and handled during the parsing and scoring phases of the tool.\end{Desc}
\begin{Desc}
\item[EXP\_\-OP\_\-SIG]A signal expression has no left or right child (they are both NULL). Its vector value is a pointer to the signal vector value to which is belongs. This allows the signal expression value to change automatically when the signal value is updated. No further expression operation is necessary to calculate its value.\end{Desc}
\begin{Desc}
\item[EXP\_\-OP\_\-SBIT\_\-SEL]A single-bit signal expression has its left child pointed to the expression tree that is required to select the bit from the specified signal value. The left child is allowed to change values during simulation. To verify that the current bit select has not exceeded the ranges of the signal, the signal pointer value in the expression structure is used to reference the signal. The LSB and width values from the actual signal can then be used to verify that we are still within range. If we are found to be out of range, a value of X must be assigned the the SBIT\_\-SEL expression. The width of an SBIT\_\-SEL is always constant (1). The LSB of the SBIT\_\-SEL is manipulated by the left expression value.\end{Desc}
\begin{Desc}
\item[EXP\_\-OP\_\-MBIT\_\-SEL]A multi-bit signal expression has its left child set to the expression tree on the left side of the ':' in the vector and the right child set to the expression tree on the right side of the ':' in the vector. The width of the MBIT\_\-SEL must be constant but is related to the difference between the left and right child values; therefore, it is required that the left and right child values be constant expressions (consisting of only expressions, parameters, and static values). The width of the MBIT\_\-SEL expression is calculated after reading in the MBIT\_\-SEL expression from the CDD file. If the left or right child expressions are found to not be constant, an error is signaled to the user immediately. The LSB is also calculated to be the lesser of the two child values. The width and lsb are assigned to the MBIT\_\-SEL expression vector immediately. In the case of MBIT\_\-SEL, the LSB is also constant. Vector direction is currently not considered at this point.\end{Desc}


\subsection{Function Documentation}
\index{expr.c@{expr.c}!expression_bit_value@{expression\_\-bit\_\-value}}
\index{expression_bit_value@{expression\_\-bit\_\-value}!expr.c@{expr.c}}
\subsubsection{\setlength{\rightskip}{0pt plus 5cm}int expression\_\-bit\_\-value ({\bf expression} $\ast$ {\em expr})}\label{expr_8c_a21}


Returns a compressed, 1-bit representation of the value after a unary OR.

\begin{Desc}
\item[Parameters:]
\begin{description}
\item[{\em expr}]Pointer to expression to evaluate.\end{description}
\end{Desc}
\begin{Desc}
\item[Returns:]Returns the value of the expression after being compressed to 1 bit via a unary OR.\end{Desc}
Returns a value of 1 if the specified expression contains at least one 1 value and no X or Z values in its bits. It accomplishes this by performing a unary OR operation on the specified expression value and testing bit 0 of the result. \index{expr.c@{expr.c}!expression_create@{expression\_\-create}}
\index{expression_create@{expression\_\-create}!expr.c@{expr.c}}
\subsubsection{\setlength{\rightskip}{0pt plus 5cm}{\bf expression}$\ast$ expression\_\-create ({\bf expression} $\ast$ {\em right}, {\bf expression} $\ast$ {\em left}, int {\em op}, int {\em id}, int {\em line}, {\bf bool} {\em data})}\label{expr_8c_a11}


Creates new expression.

\begin{Desc}
\item[Parameters:]
\begin{description}
\item[{\em right}]Pointer to expression on right. \item[{\em left}]Pointer to expression on left. \item[{\em op}]Operation to perform for this expression. \item[{\em id}]ID for this expression as determined by the parent. \item[{\em line}]Line number this expression is on. \item[{\em data}]Specifies if we should create a nibble array for the vector value.\end{description}
\end{Desc}
\begin{Desc}
\item[Returns:]Returns pointer to newly created expression.\end{Desc}
Creates a new expression from heap memory and initializes its values for usage. Right and left expressions need to be created before this function is called. \index{expr.c@{expr.c}!expression_create_value@{expression\_\-create\_\-value}}
\index{expression_create_value@{expression\_\-create\_\-value}!expr.c@{expr.c}}
\subsubsection{\setlength{\rightskip}{0pt plus 5cm}void expression\_\-create\_\-value ({\bf expression} $\ast$ {\em exp}, int {\em width}, int {\em lsb}, {\bf bool} {\em data})}\label{expr_8c_a10}


Creates an expression value and initializes it.

\begin{Desc}
\item[Parameters:]
\begin{description}
\item[{\em exp}]Pointer to expression to add value to. \item[{\em width}]Width of value to create. \item[{\em lsb}]Least significant value of value field. \item[{\em data}]Specifies if nibble array should be allocated for vector.\end{description}
\end{Desc}
Creates a value vector that is large enough to store width number of bits in value and sets the specified expression value to this value. This function should be called by either the expression\_\-create function, the bind function, or the signal db\_\-read function. \index{expr.c@{expr.c}!expression_db_merge@{expression\_\-db\_\-merge}}
\index{expression_db_merge@{expression\_\-db\_\-merge}!expr.c@{expr.c}}
\subsubsection{\setlength{\rightskip}{0pt plus 5cm}{\bf bool} expression\_\-db\_\-merge ({\bf expression} $\ast$ {\em base}, char $\ast$$\ast$ {\em line}, {\bf bool} {\em same})}\label{expr_8c_a17}


Reads and merges two expressions and stores result in base expression.

\begin{Desc}
\item[Parameters:]
\begin{description}
\item[{\em base}]Expression to merge data into. \item[{\em line}]Pointer to CDD line to parse. \item[{\em same}]Specifies if expression to be merged needs to be exactly the same as the existing expression.\end{description}
\end{Desc}
\begin{Desc}
\item[Returns:]Returns TRUE if parse and merge was sucessful; otherwise, returns FALSE.\end{Desc}
Parses specified line for expression information and merges contents into the base expression. If the two expressions given are not the same (IDs, op, and/or line position differ) we know that the database files being merged were not created from the same design; therefore, display an error message to the user in this case. If both expressions are the same, perform the merge. \index{expr.c@{expr.c}!expression_db_read@{expression\_\-db\_\-read}}
\index{expression_db_read@{expression\_\-db\_\-read}!expr.c@{expr.c}}
\subsubsection{\setlength{\rightskip}{0pt plus 5cm}{\bf bool} expression\_\-db\_\-read (char $\ast$$\ast$ {\em line}, {\bf module} $\ast$ {\em curr\_\-mod}, {\bf bool} {\em eval})}\label{expr_8c_a16}


Reads current line of specified file and parses for expression information.

\begin{Desc}
\item[Parameters:]
\begin{description}
\item[{\em line}]String containing database line to read information from. \item[{\em curr\_\-mod}]Pointer to current module that instantiates this expression. \item[{\em eval}]If TRUE, evaluate expression if children are static.\end{description}
\end{Desc}
\begin{Desc}
\item[Returns:]Returns TRUE if parsing successful; otherwise, returns FALSE.\end{Desc}
Reads in the specified expression information, creates new expression from heap, populates the expression with specified information from file and returns that value in the specified expression pointer. If all is successful, returns TRUE; otherwise, returns FALSE. \index{expr.c@{expr.c}!expression_db_write@{expression\_\-db\_\-write}}
\index{expression_db_write@{expression\_\-db\_\-write}!expr.c@{expr.c}}
\subsubsection{\setlength{\rightskip}{0pt plus 5cm}void expression\_\-db\_\-write ({\bf expression} $\ast$ {\em expr}, FILE $\ast$ {\em file}, char $\ast$ {\em scope})}\label{expr_8c_a15}


Writes this expression to the specified database file.

\begin{Desc}
\item[Parameters:]
\begin{description}
\item[{\em expr}]Pointer to expression to write to database file. \item[{\em file}]Pointer to database file to write to. \item[{\em scope}]Name of Verilog hierarchical scope for this expression.\end{description}
\end{Desc}
This function recursively displays the expression information for the specified expression tree to the coverage database specified by file. \index{expr.c@{expr.c}!expression_dealloc@{expression\_\-dealloc}}
\index{expression_dealloc@{expression\_\-dealloc}!expr.c@{expr.c}}
\subsubsection{\setlength{\rightskip}{0pt plus 5cm}void expression\_\-dealloc ({\bf expression} $\ast$ {\em expr}, {\bf bool} {\em exp\_\-only})}\label{expr_8c_a23}


Deallocates memory used for expression.

\begin{Desc}
\item[Parameters:]
\begin{description}
\item[{\em expr}]Pointer to root expression to deallocate. \item[{\em exp\_\-only}]Removes only the specified expression and not its children.\end{description}
\end{Desc}
Deallocates all heap memory allocated with the malloc routine. \index{expr.c@{expr.c}!expression_display@{expression\_\-display}}
\index{expression_display@{expression\_\-display}!expr.c@{expr.c}}
\subsubsection{\setlength{\rightskip}{0pt plus 5cm}void expression\_\-display ({\bf expression} $\ast$ {\em expr})}\label{expr_8c_a18}


Displays the specified expression information.

\begin{Desc}
\item[Parameters:]
\begin{description}
\item[{\em expr}]Pointer to expression to display.\end{description}
\end{Desc}
Displays contents of the specified expression to standard output. This function is called by the module\_\-display function. \index{expr.c@{expr.c}!expression_get_id@{expression\_\-get\_\-id}}
\index{expression_get_id@{expression\_\-get\_\-id}!expr.c@{expr.c}}
\subsubsection{\setlength{\rightskip}{0pt plus 5cm}int expression\_\-get\_\-id ({\bf expression} $\ast$ {\em expr})}\label{expr_8c_a14}


Returns expression ID of this expression.

\begin{Desc}
\item[Parameters:]
\begin{description}
\item[{\em expr}]Pointer to expression to get ID from. \end{description}
\end{Desc}
\begin{Desc}
\item[Returns:]Returns expression ID for this expression.\end{Desc}
If specified expression is non-NULL, return expression ID of this expression; otherwise, return a value of 0 to indicate that this is a leaf node. \index{expr.c@{expr.c}!expression_is_static_only@{expression\_\-is\_\-static\_\-only}}
\index{expression_is_static_only@{expression\_\-is\_\-static\_\-only}!expr.c@{expr.c}}
\subsubsection{\setlength{\rightskip}{0pt plus 5cm}{\bf bool} expression\_\-is\_\-static\_\-only ({\bf expression} $\ast$ {\em expr})}\label{expr_8c_a22}


Returns TRUE if specified expression is found to contain all static leaf expressions.

\begin{Desc}
\item[Parameters:]
\begin{description}
\item[{\em expr}]Pointer to expression to evaluate.\end{description}
\end{Desc}
\begin{Desc}
\item[Returns:]Returns TRUE if expression contains only static expressions; otherwise, returns FALSE.\end{Desc}
Recursively iterates through specified expression tree and returns TRUE if all of the children expressions are static expressions (STATIC or parameters). \index{expr.c@{expr.c}!expression_operate@{expression\_\-operate}}
\index{expression_operate@{expression\_\-operate}!expr.c@{expr.c}}
\subsubsection{\setlength{\rightskip}{0pt plus 5cm}void expression\_\-operate ({\bf expression} $\ast$ {\em expr})}\label{expr_8c_a19}


Performs operation specified by parameter expression.

\begin{Desc}
\item[Parameters:]
\begin{description}
\item[{\em expr}]Pointer to expression to set value to.\end{description}
\end{Desc}
Performs expression operation. This function must only be run after its left and right expressions have been calculated during this clock period. Sets the value of the operation in its own vector value and updates the suppl nibble as necessary. \index{expr.c@{expr.c}!expression_operate_recursively@{expression\_\-operate\_\-recursively}}
\index{expression_operate_recursively@{expression\_\-operate\_\-recursively}!expr.c@{expr.c}}
\subsubsection{\setlength{\rightskip}{0pt plus 5cm}void expression\_\-operate\_\-recursively ({\bf expression} $\ast$ {\em expr})}\label{expr_8c_a20}


Performs recursive expression operation (parse mode only).

\begin{Desc}
\item[Parameters:]
\begin{description}
\item[{\em expr}]Pointer to top of expression tree to perform recursive operations.\end{description}
\end{Desc}
Recursively performs the proper operations to cause the top-level expression to be set to a value. This function is called during the parse stage to derive pre-CDD widths of multi-bit expressions. Each MSB/LSB is an expression tree that needs to be evaluated to set the width properly on the MBIT\_\-SEL expression. \index{expr.c@{expr.c}!expression_resize@{expression\_\-resize}}
\index{expression_resize@{expression\_\-resize}!expr.c@{expr.c}}
\subsubsection{\setlength{\rightskip}{0pt plus 5cm}void expression\_\-resize ({\bf expression} $\ast$ {\em expr}, {\bf bool} {\em recursive})}\label{expr_8c_a13}


Recursively resizes specified expression tree leaf node.

\begin{Desc}
\item[Parameters:]
\begin{description}
\item[{\em expr}]Pointer to expression to potentially resize. \item[{\em recursive}]Specifies if we should perform a recursive depth-first resize\end{description}
\end{Desc}
Resizes the given expression depending on the expression operation and its children's sizes. If recursive is TRUE, performs the resize in a depth-first fashion, resizing the children before resizing the current expression. If recursive is FALSE, only the given expression is evaluated and resized. \index{expr.c@{expr.c}!expression_set_value@{expression\_\-set\_\-value}}
\index{expression_set_value@{expression\_\-set\_\-value}!expr.c@{expr.c}}
\subsubsection{\setlength{\rightskip}{0pt plus 5cm}void expression\_\-set\_\-value ({\bf expression} $\ast$ {\em exp}, {\bf vector} $\ast$ {\em vec})}\label{expr_8c_a12}


Sets the specified expression value to the specified vector value.

\begin{Desc}
\item[Parameters:]
\begin{description}
\item[{\em exp}]Pointer to expression to set value to. \item[{\em vec}]Pointer to vector value to set expression to.\end{description}
\end{Desc}
Sets the specified expression (if necessary) to the value of the specified vector value. 

\subsection{Variable Documentation}
\index{expr.c@{expr.c}!and_optab@{and\_\-optab}}
\index{and_optab@{and\_\-optab}!expr.c@{expr.c}}
\subsubsection{\setlength{\rightskip}{0pt plus 5cm}{\bf nibble} and\_\-optab[16] ()}\label{expr_8c_a1}


AND operation table \index{expr.c@{expr.c}!curr_sim_time@{curr\_\-sim\_\-time}}
\index{curr_sim_time@{curr\_\-sim\_\-time}!expr.c@{expr.c}}
\subsubsection{\setlength{\rightskip}{0pt plus 5cm}int curr\_\-sim\_\-time ()}\label{expr_8c_a6}


This static value contains the current simulation time which is specified by the db\_\-do\_\-timestep function. It is used for calculating delay expressions in the simulation engine. \index{expr.c@{expr.c}!nand_optab@{nand\_\-optab}}
\index{nand_optab@{nand\_\-optab}!expr.c@{expr.c}}
\subsubsection{\setlength{\rightskip}{0pt plus 5cm}{\bf nibble} nand\_\-optab[16] ()}\label{expr_8c_a3}


NAND operation table \index{expr.c@{expr.c}!nor_optab@{nor\_\-optab}}
\index{nor_optab@{nor\_\-optab}!expr.c@{expr.c}}
\subsubsection{\setlength{\rightskip}{0pt plus 5cm}{\bf nibble} nor\_\-optab[16] ()}\label{expr_8c_a4}


NOR operation table \index{expr.c@{expr.c}!nxor_optab@{nxor\_\-optab}}
\index{nxor_optab@{nxor\_\-optab}!expr.c@{expr.c}}
\subsubsection{\setlength{\rightskip}{0pt plus 5cm}{\bf nibble} nxor\_\-optab[16] ()}\label{expr_8c_a5}


NXOR operation table \index{expr.c@{expr.c}!or_optab@{or\_\-optab}}
\index{or_optab@{or\_\-optab}!expr.c@{expr.c}}
\subsubsection{\setlength{\rightskip}{0pt plus 5cm}{\bf nibble} or\_\-optab[16] ()}\label{expr_8c_a2}


OR operation table \index{expr.c@{expr.c}!static_expr_head@{static\_\-expr\_\-head}}
\index{static_expr_head@{static\_\-expr\_\-head}!expr.c@{expr.c}}
\subsubsection{\setlength{\rightskip}{0pt plus 5cm}{\bf exp\_\-link}$\ast$ static\_\-expr\_\-head}\label{expr_8c_a8}


\index{expr.c@{expr.c}!static_expr_tail@{static\_\-expr\_\-tail}}
\index{static_expr_tail@{static\_\-expr\_\-tail}!expr.c@{expr.c}}
\subsubsection{\setlength{\rightskip}{0pt plus 5cm}{\bf exp\_\-link}$\ast$ static\_\-expr\_\-tail}\label{expr_8c_a9}


\index{expr.c@{expr.c}!user_msg@{user\_\-msg}}
\index{user_msg@{user\_\-msg}!expr.c@{expr.c}}
\subsubsection{\setlength{\rightskip}{0pt plus 5cm}char user\_\-msg[USER\_\-MSG\_\-LENGTH] ()}\label{expr_8c_a7}


Holds some output that will be displayed via the print\_\-output command. This is created globally so that memory does not need to be reallocated for each function that wishes to use it. \index{expr.c@{expr.c}!xor_optab@{xor\_\-optab}}
\index{xor_optab@{xor\_\-optab}!expr.c@{expr.c}}
\subsubsection{\setlength{\rightskip}{0pt plus 5cm}{\bf nibble} xor\_\-optab[16] ()}\label{expr_8c_a0}


XOR operation table 