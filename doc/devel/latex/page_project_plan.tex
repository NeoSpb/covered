\section{Section 2.  Project Plan}\label{page_project_plan}
\begin{Desc}
\item[Section 2.1. Project Goals for Usability]The goals of the Covered project as it pertains to its users are as follows:\end{Desc}
\begin{Desc}
\item[]\begin{enumerate}
\item Have the ability to parse all legal Verilog code as defined by the Verilog 2000 LRM\item Generate concise, human-readable summary reports for line, toggle, combinational, and FSM state and arc coverage in which a user may quickly discern the amount of coverage achieved.\item Generate concise, human-readable verbose reports for line, toggle, combinational, and FSM state and arc coverage in which a user may easily discern why certain coverage results did not achieve 100\% coverage. Verbose reports should contain all information necessary for diagnosing the cause for lack of coverage without being excessive in the amount of information provided to aid in readability.\item Allow the ability to easily merge generated Coverage Description Database (CDD) files to allow users to accumulate coverage results for a particular design.\item Have the ability to parse designs, generate coverage results, and generate reports via command-line calls and through the use of a GUI.\item Provide sufficient user documentation to understand how to use the tool and understand its output.\item Provide users with additional ways to get questions answered and submit bug reports/ enhancement requests via the following mechanisms: FAQ, bug reporting facility, mailing list, user manual, and Homepage \char`\"{}News\char`\"{} section.\end{enumerate}
\end{Desc}




\begin{Desc}
\item[Section 2.2. Project Goals for Development]The goals of the Covered project as it pertains to ease in development are as follows:\end{Desc}
\begin{Desc}
\item[]\begin{enumerate}
\item Write source code using C language using standard C libraries for cross-platform compatibility purposes.\item Maintain sufficient amount of in-line documentation to understand purpose of functions, structures, defines, variables, etc.\item Use autoconf and automake to generate configuration files and Makefiles that will be able to compile the source code on any UNIX-based operating system.\item Use CVS for project management and file revision purposes, allowing outside developers to contribute to source code.\item Use the Doxygen utility for generating documentation from source files that is used for development reference.\item Develop self-checking, self-contained diagnostic regression suite to verify new and existing features of code to assure that new releases are backwards compatible to older versions of the tool and that new features have been tested adequately prior to tool releases to the general public.\item Provide sufficient development documentation to allow new and existing developer's to understand how Covered works, the procedures/practices used in the development process and other development-specific information.\item Provide developer's with a means of communicating project ideas, status or other announcements to other project developers via the following mechanisms: CVS, mailing lists, and bug reporting facility.\end{enumerate}
\end{Desc}




\begin{Desc}
\item[Section 2.3. Project Goals for Distribution]The goals of the Covered project as it pertains to ease in project releases and distributions are as follows:\end{Desc}
\begin{Desc}
\item[]\begin{enumerate}
\item Provide source code in tarball format (tar'ed and gzip'ed) which will be accessible via the Covered homepage.\item Provide any links to RPMs, Debian packages, etc. that others provide for the project.\item Generate new releases/information on a timely basis so that users of the tool do not question whether development is still occurring with the project.\item Generate stable releases for users.\item Generate development releases in CVS for branching and regression purposes.\item Verify that user documentation does not become stale but rather is synchronized with the current release.\end{enumerate}
\end{Desc}




\begin{Desc}
\item[Go To Section...]\begin{itemize}
\item {\bf Section 1.  Introduction} \item {\bf Section 3.  Coding Style Guidelines} \item {\bf Section 4.  Development Tools} \item {\bf Section 5.  Project \char`\"{}Big Picture\char`\"{}} \item {\bf Section 6.  Coverage Development Reference} \item {\bf Section 7.  Test and Checkout Procedure} \item {\bf Section 8.  Odds and Ends Information} \end{itemize}
\end{Desc}
