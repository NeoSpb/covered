\section{module.h File Reference}
\label{module_8h}\index{module.h@{module.h}}
Contains functions for handling modules.  


{\tt \#include $<$stdio.h$>$}\par
{\tt \#include \char`\"{}defines.h\char`\"{}}\par
\subsection*{Functions}
\begin{CompactItemize}
\item 
void {\bf module\_\-init} ({\bf module} $\ast$mod)
\begin{CompactList}\small\item\em Initializes all values of module. \item\end{CompactList}\item 
{\bf module} $\ast$ {\bf module\_\-create} ()
\begin{CompactList}\small\item\em Creates new module from heap and initializes structure. \item\end{CompactList}\item 
{\bf bool} {\bf module\_\-db\_\-write} ({\bf module} $\ast$mod, char $\ast$scope, FILE $\ast${\bf file}, {\bf mod\_\-inst} $\ast$inst)
\begin{CompactList}\small\item\em Writes contents of provided module to specified output. \item\end{CompactList}\item 
{\bf bool} {\bf module\_\-db\_\-read} ({\bf module} $\ast$mod, char $\ast$scope, char $\ast$$\ast$line)
\begin{CompactList}\small\item\em Read contents of current line from specified file, creates module and adds to module list. \item\end{CompactList}\item 
{\bf bool} {\bf module\_\-db\_\-merge} ({\bf module} $\ast$base, FILE $\ast${\bf file}, {\bf bool} same)
\begin{CompactList}\small\item\em Reads and merges two modules into base module. \item\end{CompactList}\item 
void {\bf module\_\-display\_\-signals} ({\bf module} $\ast$mod)
\begin{CompactList}\small\item\em Displays signals stored in this module. \item\end{CompactList}\item 
void {\bf module\_\-display\_\-expressions} ({\bf module} $\ast$mod)
\begin{CompactList}\small\item\em Displays expressions stored in this module. \item\end{CompactList}\item 
void {\bf module\_\-clean} ({\bf module} $\ast$mod)
\begin{CompactList}\small\item\em Deallocates module element contents only from heap. \item\end{CompactList}\item 
void {\bf module\_\-dealloc} ({\bf module} $\ast$mod)
\begin{CompactList}\small\item\em Deallocates module element from heap. \item\end{CompactList}\end{CompactItemize}


\subsection{Detailed Description}
Contains functions for handling modules. 

\begin{Desc}
\item[Author:]Trevor Williams ({\tt trevorw@charter.net}) \end{Desc}
\begin{Desc}
\item[Date:]12/7/2001 \end{Desc}


\subsection{Function Documentation}
\index{module.h@{module.h}!module_clean@{module\_\-clean}}
\index{module_clean@{module\_\-clean}!module.h@{module.h}}
\subsubsection{\setlength{\rightskip}{0pt plus 5cm}void module\_\-clean ({\bf module} $\ast$ {\em mod})}\label{module_8h_a7}


Deallocates module element contents only from heap. 

\begin{Desc}
\item[Parameters:]
\begin{description}
\item[{\em mod}]Pointer to module element to clean.\end{description}
\end{Desc}
Deallocates module contents: name and filename strings. \index{module.h@{module.h}!module_create@{module\_\-create}}
\index{module_create@{module\_\-create}!module.h@{module.h}}
\subsubsection{\setlength{\rightskip}{0pt plus 5cm}{\bf module}$\ast$ module\_\-create ()}\label{module_8h_a1}


Creates new module from heap and initializes structure. 

\begin{Desc}
\item[Returns:]Returns pointer to newly created module element that has been properly initialized.\end{Desc}
Allocates memory from the heap for a module element and initializes all contents to NULL. Returns a pointer to the newly created module. \index{module.h@{module.h}!module_db_merge@{module\_\-db\_\-merge}}
\index{module_db_merge@{module\_\-db\_\-merge}!module.h@{module.h}}
\subsubsection{\setlength{\rightskip}{0pt plus 5cm}{\bf bool} module\_\-db\_\-merge ({\bf module} $\ast$ {\em base}, FILE $\ast$ {\em file}, {\bf bool} {\em same})}\label{module_8h_a4}


Reads and merges two modules into base module. 

\begin{Desc}
\item[Parameters:]
\begin{description}
\item[{\em base}]Module that will merge in that data from the in module \item[{\em file}]Pointer to CDD file handle to read. \item[{\em same}]Specifies if module to be merged should match existing module exactly or not.\end{description}
\end{Desc}
\begin{Desc}
\item[Returns:]Returns TRUE if parse and merge was successful; otherwise, returns FALSE.\end{Desc}
Parses specified line for module information and performs a merge of the two specified modules, placing the resulting merge module into the module named base. If there are any differences between the two modules, a warning or error will be displayed to the user. \index{module.h@{module.h}!module_db_read@{module\_\-db\_\-read}}
\index{module_db_read@{module\_\-db\_\-read}!module.h@{module.h}}
\subsubsection{\setlength{\rightskip}{0pt plus 5cm}{\bf bool} module\_\-db\_\-read ({\bf module} $\ast$ {\em mod}, char $\ast$ {\em scope}, char $\ast$$\ast$ {\em line})}\label{module_8h_a3}


Read contents of current line from specified file, creates module and adds to module list. 

\begin{Desc}
\item[Parameters:]
\begin{description}
\item[{\em mod}]Pointer to module to read contents into. \item[{\em scope}]Pointer to name of read module scope. \item[{\em line}]Pointer to current line to parse.\end{description}
\end{Desc}
\begin{Desc}
\item[Returns:]Returns TRUE if read was successful; otherwise, returns FALSE.\end{Desc}
Reads the current line of the specified file and parses it for a module. If all is successful, returns TRUE; otherwise, returns FALSE. \index{module.h@{module.h}!module_db_write@{module\_\-db\_\-write}}
\index{module_db_write@{module\_\-db\_\-write}!module.h@{module.h}}
\subsubsection{\setlength{\rightskip}{0pt plus 5cm}{\bf bool} module\_\-db\_\-write ({\bf module} $\ast$ {\em mod}, char $\ast$ {\em scope}, FILE $\ast$ {\em file}, {\bf mod\_\-inst} $\ast$ {\em inst})}\label{module_8h_a2}


Writes contents of provided module to specified output. 

\begin{Desc}
\item[Parameters:]
\begin{description}
\item[{\em mod}]Pointer to module to write to output. \item[{\em scope}]String version of module scope in hierarchy. \item[{\em file}]Pointer to specified output file to write contents. \item[{\em inst}]Pointer to the current module instance.\end{description}
\end{Desc}
\begin{Desc}
\item[Returns:]Returns TRUE if file output was successful; otherwise, returns FALSE.\end{Desc}
Prints the database line for the specified module to the specified database file. If there are any problems with the write, returns FALSE; otherwise, returns TRUE. \index{module.h@{module.h}!module_dealloc@{module\_\-dealloc}}
\index{module_dealloc@{module\_\-dealloc}!module.h@{module.h}}
\subsubsection{\setlength{\rightskip}{0pt plus 5cm}void module\_\-dealloc ({\bf module} $\ast$ {\em mod})}\label{module_8h_a8}


Deallocates module element from heap. 

\begin{Desc}
\item[Parameters:]
\begin{description}
\item[{\em mod}]Pointer to module element to deallocate.\end{description}
\end{Desc}
Deallocates module; name and filename strings; and finally the structure itself from the heap. \index{module.h@{module.h}!module_display_expressions@{module\_\-display\_\-expressions}}
\index{module_display_expressions@{module\_\-display\_\-expressions}!module.h@{module.h}}
\subsubsection{\setlength{\rightskip}{0pt plus 5cm}void module\_\-display\_\-expressions ({\bf module} $\ast$ {\em mod})}\label{module_8h_a6}


Displays expressions stored in this module. 

\begin{Desc}
\item[Parameters:]
\begin{description}
\item[{\em mod}]Pointer to module element to display expressions\end{description}
\end{Desc}
Iterates through expression list of specified module, displaying each expression's id. \index{module.h@{module.h}!module_display_signals@{module\_\-display\_\-signals}}
\index{module_display_signals@{module\_\-display\_\-signals}!module.h@{module.h}}
\subsubsection{\setlength{\rightskip}{0pt plus 5cm}void module\_\-display\_\-signals ({\bf module} $\ast$ {\em mod})}\label{module_8h_a5}


Displays signals stored in this module. 

\begin{Desc}
\item[Parameters:]
\begin{description}
\item[{\em mod}]Pointer to module element to display signals.\end{description}
\end{Desc}
Iterates through signal list of specified module, displaying each signal's name, width, lsb and value. \index{module.h@{module.h}!module_init@{module\_\-init}}
\index{module_init@{module\_\-init}!module.h@{module.h}}
\subsubsection{\setlength{\rightskip}{0pt plus 5cm}void module\_\-init ({\bf module} $\ast$ {\em mod})}\label{module_8h_a0}


Initializes all values of module. 

\begin{Desc}
\item[Parameters:]
\begin{description}
\item[{\em mod}]Pointer to module to initialize.\end{description}
\end{Desc}
Initializes all contents to NULL. 