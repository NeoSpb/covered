\section{Section 1.  Introduction}\label{page_intro}
\begin{Desc}
\item[]This documentation is specific to the development of the Covered tool. For usage-specific information, please consult the Covered User's Guide which is accessible via tarball download or off of the Covered hompage.\end{Desc}
\begin{Desc}
\item[]Welcome to Covered development! Since you are reading this document, it is assumed that you are either on the development team, looking to join the team, or are just interested in some of the \char`\"{}under the hood\char`\"{} technical details about how Covered is intended to get the job done. This document will seek to specify the overall project development plan; coding methodologies; project communication guidelines; programs/utilities needed for code development and documentation generation; the code's \char`\"{}big picture\char`\"{}; the nitty-gritty details for all structures, functions and defines; testing procedure; and some odds and ends information. This document will serve as a technical reference as well as the \char`\"{}Covered constitution\char`\"{} on programming guidelines for this project.\end{Desc}
\begin{Desc}
\item[]But first of all, what is the purpose of this project? Covered is a Verilog code coverage analyzation utility that allows a user to examine the effectiveness of a suite of diagnostics testing a design-under-test (DUT). {\bf TBD}  \end{Desc}




\begin{Desc}
\item[Go To Section...]\begin{itemize}
\item {\bf Section 2.  Project Plan} \item {\bf Section 3.  Coding Style Guidelines} \item {\bf Section 4.  Development Tools} \item {\bf Section 5.  Project \char`\"{}Big Picture\char`\"{}} \item {\bf Section 6.  Coverage Development Reference} \item {\bf Section 7.  Test and Checkout Procedure} \item {\bf Section 8.  Odds and Ends Information} \end{itemize}
\end{Desc}
